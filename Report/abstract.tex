In this report, we study the behaviour of the simple pendulum, as well as the numerical relations between the parameters of the function which governs its motion. A pendulum consisting of two strings supporting a weight to two independently fixed pivot points,  designed in such a way as to allow the mass, length, and amplitude of the system to easily be altered, is used to collect all the data considered in this report. It is concluded that the angular amplitude depends on time periodically with an exponentially decaying envelope so long as the initial angular amplitude lies in the domain of $|\theta_0| < 0.25\text{rads}$. It is further concluded that the length of the pendulum dictates the observed period of oscillation as $T = 2\sqrt{\ell}$, where $T$ is the period of oscillation and $\ell$ is the length of the pendulum. The decay coefficient, $\tau$ is observed to have quadratic relations depending on both the length and initial angular amplitude, while having logarithmic dependence on the mass $m$ of the weight being used. Finally, a numerical estimation of the symmetry is underwent to conclude that the constructed pendulum has 98.7\% symmetry, and by extension, implies strong correlation between the expected behaviour and the observed behaviour of the system.
