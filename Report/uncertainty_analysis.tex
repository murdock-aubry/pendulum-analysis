Multiple sources of uncertainty come into play in the numerical analysis of 
this pendulum. Minor contributions arise from deviations in the length of 
the string throughout the motion of the mass. As to ensure a sufficiently 
large uncertainty is propagated for all combinations of the independent variables 
(in particular paying attention to the proportionality between the magnitude of 
the stretching and the mass of the object), it was measured that the string 
stretched a distance $2.00\text{mm}$ between a $20.00\text{g}$ mass (as to keep 
the string taught) and the $500.00\text{g}$ mass (the largest mass used), and 
therefore an uncertainty value of $2.00\text{m}$ is placed on the length of 
the string $L$. Other uncertainties may arise from motion of the pivot point, 
however the system used two large $30\text{kg}$ stands with broad bases, and 
no motion of either pivot point was visible throughout the experiment. As such, 
no additional uncertainty is propagated from this plausible source of error.\\[0.10cm]

The largest source of uncertainty, however, arises from within the tracking 
methods themselves. As briefly discussed in the experimental procedure section 
of the report, Tracker pinpoints a \emph{single} pixel on each frame, rather 
then an entire object. Efforts to reduce this source of uncertainty were deployed 
by placing a small piece of tape at the centre of mass (from the perspective of the 
camera), however the size of the tape itself is certainly non-negligible. The same 
size of tape is used in each set of data, the diameter of which is $0.006\text{m}$.
 It is assumed that the geometric centre of the tape is recorded on each frame, 
 thus resulting in an uncertainty value of $0.003\text{m}$ in both the $x$ and $y$ 
 coordinates. Using this value, the uncertainty in the angle $\theta$ is propagated 
 using Eq.(\ref{eq: uncertainty}). Given values of $x$ and $y$, the angle $\theta$ 
 is then given by $\theta(x, y) = \ell \arctan\left(\frac{x}{y} \right)$, so that 
 the uncertainty in $\theta$ is given by;

\begin{equation} \label{eq: theta_unc}
    \sigma_\theta = \ell\sqrt{\frac{x^2\sigma_x^2 + y^2\sigma_y^2}{(x^2 + y^2)^2}} = \ell\sqrt{\frac{(0.003\text{m})^2}{x^2 + y^2}}
\end{equation}

Note that the uncertainty in the value $\ell$ here is ignored, and such an 
approximation is justified since the only term involving such is $\sigma_\ell^2/\ell^2 \approx 0$. 
The error is applied to every data point in every data set, and has the largest 
magnitude over all other uncertainties making the tracking methods the largest 
source of uncertainty.\\[0.20cm]

