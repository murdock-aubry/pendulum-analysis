The goal here is to provide a numerical domain of the initial angular amplitude for which the theory discussed in the introduction. Issues resulting from a large angular amplitude appear in the expected independence of the period of oscillation with respect to the initial angular amplitude, as well as for the validity of the functional fit of the form Eq.(\ref{theta equation}) for large initial angles. As discussed above, consistency in the estimated period (up to uncertainty) remains constant for various initial angles so long as the magnitude of the initial angle does not exceed $0.25\text{rads}$. Conversely, the goodness of fit estimations for Figures (\ref{P1}) through (\ref{P6}) in the appendix remain reasonable up to Figure (\ref{P5}), at which point the functional form does \emph{not} accurately represent the behaviour of the data. It is thus evident that the behaviour of the pendulum follows the form of Eq.(\ref{theta equation}) for cases where the initial angular amplitude does not exceed $0.41\text{rads}$ (which was the initial angular amplitude of the data plotted in Figure (\ref{P4})). Taking the stricter of the two (independently obtained) conditions, a small angle approximation is valid for values of $\theta$ which lie in the domain $|\theta_0| < 0.25\text{rads}$.\\[0.20cm]

