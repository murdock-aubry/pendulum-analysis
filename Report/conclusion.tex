As hypothesized in the introduction, the angular amplitude of the pendulum is 
governed by Eq.(\ref{theta equation}) so long as $|\theta_0| < 0.25\text{rads}$; 
the domain at which the small angle approximation remains valid. It is further 
concluded that both the mass and the angular amplitude have no impact on period 
of oscillation of the pendulum, so long as the small angle approximation is 
uniformly satisfied. A quantitative analysis of the system results in an 
estimated symmetry of $98.7\%$ and a high level of mechanical symmetry, 
obtaining an average decay coefficient to period of oscillation quotient of 
108.02. High efficiency and symmetry lead directly to minimal uncertainties and 
by extension, accurate results which align with the theoretical framework 
introduced in the introduction. Moreover, through varying the length of the 
pendulum $\ell$ (and leaving all other variables fixed) it is concluded that 
the period of oscillation $T$ depends on $\ell$ as $T = 2\sqrt{\ell}$. Finally, 
the drag coefficient estimations yields data that indicates quadratically 
increasing, logarithmic, and quadratically decreasing dependence of $\tau$ on 
the length $\ell$, mass $m$, and initial amplitude $\theta_0$ respectively.
