A perfectly symmetric simple pendulum with no damping due to drag would have the 
property that $\theta(t) = -\theta(t + T/2)$, for all time $t$ where $T$ is the 
period of oscillation. Any deviation from this equality implies asymmetry within 
the system. To quantify the asymmetry of the pendulum used to collect the data 
in this report, this equality is tested at each point in time $t$. The idea is 
as follows: the first 100 data points\footnote{The reasoning for only using the 
first 100 data points is because the symmetry of the system is entirely 
determined after undergoing a single full oscillation anyways. The first 100 
time steps include approximately four full oscillations of the pendulum, and 
thus this is sufficient to accurately estimate the observed level of symmetry.} 
of a single data set is considered and fitted with a function of the form of 
Eq.(\ref{theta equation}), and as such, an estimation of the period of 
oscillation over this set of data is obtained. This functional form is then 
used to obtain residuals between the functional form and the recorded data, 
values of which can then be averaged to obtain an average \textit{absolute} 
residual value. Now shift the functional form $\theta(t)$ by an amount $T/2$ 
to the left, and multiply by -1: $\theta(t) \mapsto -\theta(t + T/2)$. For a 
perfectly symmetric pendulum, the average value of absolute residuals between 
the collected data and the \emph{shifted} functional form should be precisely 
equal. Hence, the relative change between these two averages provides a 
quantitative estimate of the \emph{asymmetry} of the pendulum, and subtracting 
this value from 1 provides a quantitative estimate of the \emph{symmetry} of 
the pendulum. \\[0.20cm]

Plotted in Figures \ref{symmetry}, \ref{symmetry_residual} are the first 100 
time steps of the motion of pendulum with a length $L = 0.1\text{m}$, a mass 
of $m = 0.1\text{kg}$ and initial angular amplitude $\theta_0 = 0.22\text{rads}$ 
plotted alongside a optimized functional form, and the associated residual 
values respectively. Plotted in Figure \ref{symmetry_shifted} is the same 
collected data, only plotted alongside the same functional form, but shifted 
to the left by a time $T/2 = (0.360\pm 0.002)\text{s}$. Finally, plotted in 
Figure \ref{symmetry_shifted_residual} is the residuals of the collected data 
and the \emph{shifted} functional form. Visually, Figures \ref{symmetry} and 
\ref{symmetry_shifted} are extremely similar, and likewise for Figures 
\ref{symmetry_residual} and \ref{symmetry_shifted_residual} -- this speaks 
to the high level of symmetry present in the pendulum. In fact, up to 
uncertainty in the angular amplitude $\theta$, the pendulum is perfectly 
symmetric. With that being said, the average (absolute) residual value present 
in Figure \ref{symmetry_residual} is 0.0048, while that of Figure 
\ref{symmetry_shifted_residual} is 0.0049, providing a relative difference of 
\[C = \frac{0.0049 - 0.0048}{0.0049} = 0.013\]
In other words, the pendulum is $1.3\%$ asymmetric or equivalently, 
$98.7\%$ symmetric: a fairly high level of symmetry. Reasons for any asymmetry 
at all are attributed to the effect of drag and hence the exponential decay of 
the envelope of the collected data.\\[0.20cm]
